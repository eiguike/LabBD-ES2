\documentclass[a4paper,12pt]{article}
\usepackage[brazil]{babel}
\usepackage[utf8]{inputenc}
\usepackage[T1]{fontenc}
\usepackage{lmodern}
\usepackage[normalem]{ulem}
\usepackage{setspace}
\usepackage{indentfirst}
\usepackage{verbatim}
\usepackage{graphicx}
\usepackage{longtable}
\usepackage{url}
\usepackage[final]{pdfpages}
% Configuração da página, paragrafos e fonte
%----------------------------------------------------------------------------

\usepackage[a4paper]{geometry}

\geometry{hmargin={3cm,3cm},vmargin={3.5cm,2.5cm}} % margens horizontais com 3cm e verticais com 3.5 e 2.5

% section diferenciado
\usepackage{titlesec, color}
\definecolor{bleudefrance}{rgb}{0.19, 0.55, 0.91}
\newcommand{\hsp}{\hspace{5pt}}
\titleformat{\section}[hang]{\Large\bfseries}{\thesection\hsp\textcolor{bleudefrance}{|}\hsp}{0pt}{\Large\bfseries}
% section diferenciado


% Cores nas linhas da tabela
\usepackage{color, colortbl}
\definecolor{electricgreen}{rgb}{0.0, 1.0, 0.0}
\definecolor{ultramarineblue}{rgb}{0.25, 0.5, 0.96}



\pagestyle{headings} % numeração de página no cabeçalho
\onehalfspacing
\setlength{\parindent}{1cm} % Default is 15pt.

%%%% NOME DO DOCUMENTO NO OVERLEAF
\title{Status Report 7}

%%%% METADADOS DO PDF   
   \pdfinfo{
   /Author (Alessandro Henrique Gabriela Gustavo)
   /Title  (Status Report 7 380105 490016 489689 489999)
%    /CreationDate (D:20040502195600)
   /Subject (Integrado)
   /Keywords (ES2 LabBD Status Report)
}

\begin{document}

%Para remover os títulos do cabeçalho
%http://www.latex-community.org/forum/viewtopic.php?f=47&t=4700
\renewcommand*\sectionmark[1]{\markboth{#1}{}}
\renewcommand*\subsectionmark[1]{\markboth{#1}{}}

% ##################################
% *CAPA!
% ##################################

\begin{titlepage}
%----------------------------------------------------------------------------
\begin{center}
{\bf \large UNIVERSIDADE FEDERAL DE SÃO CARLOS}\\[0.2cm]
{\large BACHARELADO EM CIÊNCIA DA COMPUTAÇÃO}\\[0.2cm]

% {\bf \large CENTRO DE CIÊNCIAS E TECNOLOGIAS PARA A SUSTENTABILIDADE}\\[0.2cm]
% {\bf \large CAMPUS SOROCABA}\\[1cm]
\end{center}

\vfill
\begin{center}
{\bf \large ENGENHARIA DE SOFTWARE II E\\LABORATÓRIO DE BANCO DE DADOS}\\[3.2cm]
\end{center}

\begin{center}
{\bf \LARGE PROJETO INTEGRADO}\\[0.3cm]
{\bf \Large GRUPO 8 - PROGRAMA + NATUREZA DA DESPESA}\\[2.2cm]
\end{center}

\vfill
\begin{flushright}
{\large \textbf{DOCENTES}: ALEXANDRE ÁLVARO}\\[0.2cm]
{\large SAHUDY MONTENEGRO GONZÁLES}\\[0.5cm]
\end{flushright}

\vfill
\begin{flushright}
{\large {\bf ALUNOS}: ALESSANDRO VISOTTO PICCOLI: 380105}\\[0.15cm]
{\large HENRIQUE EIHARA: 490016}\\[0.15cm]
{\large GABRIELA DE JESUS MARTINS: 489689}\\[0.15cm]
{\large GUSTAVO RODRIGUES: 489999}\\[0.15cm]
%{\large VALDEIR SOARES PEROZIM: 489786}\\[1cm]
\end{flushright}

\vfill
\begin{flushright}
{\large Data de entrega: 01/06/2015}\\[0.2cm]
{\large Status Report 7}\\[2.0cm]
\end{flushright}

\begin{center}
{\large Sorocaba}\\[0.2cm]
{\large 2015}
\end{center}

\end{titlepage}
%----------------------------------------------------------------------------

% ##################################
% *CORPO DO TEXTO!
% ##################################
\newpage

\section{Resumo do Período}

\begin{longtable}{|l|l|}
\hline
\textbf{Status Report} & 7\\
\hline
\textbf{Período}	&	18/05/2015 à 01/06/2015 \\
\hline
\end{longtable}

%Finalização da otimização do Banco de Dados, além do início da implementação interna do SGDB.

\section{Atividades do Período}

Neste período foram finalizadas as eventuais correções e melhorias e os devidos testes no \textit{Control} e também no banco de dados.


\textbf{É provavelmente o Status Report mais importante até o momento, pois foram alterados pontos relevantes na aplicação:}

Nas consultas simples, tanto por natureza ou por programa receberam mais um parâmetro além de município:

\begin{itemize}
\item Programa: recebe um programa específico;

\item Natureza: recebe uma natureza específica.
\end{itemize}

Se não for passada nenhuma natureza ou programa, será retornado os gastos de todas naturezas/programas, já se for passado o parâmetro, retorna apenas o da natureza/programa específica (se houver).

Realizamos esta alteração por dois motivos, o banco de dados tem apenas a cidade de Campinas como município, portanto, não faria sentido passar só município como parâmetro e também, porque o grupo não conseguia aplicar índice (que otimiza a consulta e é um requisito de Laboratório de Banco de Dados) na consulta anterior, e passando este novo parâmetro conseguimos.

Houve também uma consulta acrescentada ao histórico, foi criada uma tabela no banco (histórico), retornando a consulta realizada, a data em que a consulta foi realizada e o horário que foi consultada.

\textbf{Enfatizamos que estas alterações foram feitas na semana do dia 18 de maio de 2015, quando houve os testes e os ajustes finais, estas foram bem recebidas e autorizadas pela cliente Sahudy.}







\newpage
\section{Cronograma}

A seguir o cronograma com a estimativa de tempo para realizar as atividades que compõem as entregas do produto. A parte verde representa as atividades já concluídas:\\

% , a azul representa as que tiveram alterações no prazo

{\normalsize %Aplica o tamanho de fonte abaixo para toda tabela. 2ª chave está lá em baixo

\begin{longtable}{|l|c|c|}
\hline
\multicolumn{1}{|c|}{\textbf{Atividade}}  & \multicolumn{1}{c|}{\textbf{Início}} & \multicolumn{1}{c|}{\textbf{Fim}} \\ \hline

\hline
\rowcolor{electricgreen}
Definição das consultas ao banco de dados & 24/03/2015                           & 27/03/2015                        \\ \hline

\hline
\rowcolor{electricgreen}
Implementação das consultas SQL           & 27/03/2015                           & 31/03/2015                        \\ \hline

\hline
\rowcolor{electricgreen}
Desenvolvimento do plano de projeto       & 02/04/2015                           & 06/04/2015                        \\ 

\hline
\rowcolor{electricgreen}
Prototipagem das janelas de consulta      & 06/04/2015                           & 07/04/2015                        \\ \hline

\hline
\rowcolor{electricgreen}
Implementação interna do SGBD             & 24/04/2015                           & 30/04/2015                        \\ \hline

\hline
\rowcolor{electricgreen}
Implementação do \textit{Model}           & 04/05/2015                           & 11/05/2015                        \\ \hline

\hline
\rowcolor{electricgreen}
Otimização/Indexação do Banco de Dados    & 04/05/2015                           & 11/05/2015                        \\ \hline

\hline
\rowcolor{electricgreen}
Implementação da \textit{View}            & 04/05/2015                           & 18/05/2015                        \\ \hline

\hline
\rowcolor{electricgreen}
Implementação do \textit{Control}         & 09/05/2015                           & 18/05/2015                        \\ \hline

\hline
\rowcolor{electricgreen}
Teste da \textit{View}                    & 15/05/2015                           & 15/05/2015                        \\ \hline

\hline
\rowcolor{electricgreen}
Ajuste da \textit{View}                   & 15/05/2015                           & 17/05/2015                        \\ \hline

\hline
\rowcolor{electricgreen}
Teste da \textit{View} + \textit{Control} & 18/05/2015                           & 18/05/2015                        \\ \hline

\hline
\rowcolor{electricgreen}
Ajuste do \textit{Control}                & 18/05/2015                           & 19/05/2015                        \\ \hline

\hline
\rowcolor{electricgreen}
Ajustes finais no banco de dados          & 19/05/2015                           & 25/05/2015                        \\ \hline
Entrega final                             & \multicolumn{1}{c|}{-}               & 08/06/2015                        \\ \hline
Apresentação final                        & \multicolumn{1}{c|}{-}               & 10/06/2015                        \\ \hline
\end{longtable}
}


\end{document}